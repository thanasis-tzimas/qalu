Παραδείγματος χάριν, θέλουμε να υπολογίσουμε το άθροισμα $2 + 1$. Για να το καταφέρουμε
αυτό, πρώτα πρέπει να αλλάξουμε το σύστημα αρίθμησης των αριθμών εισόδου στο δυαδικό σύστημα.
$2_{10}\mapsto10_2$ και $1_{10}\mapsto01_2$. Απο αυτό το σημείο οι θεμελιώδεις κανόνες
της αριθμητικής συνεχίζουν να ισχύουν. Το μόνο που απομένει είναι να εκτέλεσουμε την πρόσθεση.
\begin{center}
    \begin{tabular}{c@{\,}c@{\,}c}
        & 1 & 0 \\
      + & 0 & 1 \\
      \hline
        & 1 & 1 \\
      \end{tabular}
\end{center}
Όπως περιμέναμε, το αποτέλεσμα της πράξης $10_2 + 01_2 = 11_2, 11_2 \mapsto 3_{10}$.
Προσέχουμε ότι στην δυαδική αριθμητική, όταν προσθέτουμε $1 + 1 = 10$ και $1 + 0 = 0 + 1 = 1$.
Γνωρίζοντας τις παραπάνω ισότητες μπορούμε να υλοποιήσουμε ένα κύκλωμα το οποίο μπορεί
να υπολογίσει το άθροισμα δυο δυαδικών αριθμών.